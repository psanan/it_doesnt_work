\documentclass{article}
\usepackage{hyperref}
\usepackage{nopageno}
\usepackage{graphicx}

\title{\textbf{It Doesn't Work!}}
\date{}
\author{}

\begin{document}
\maketitle
\vspace{-50pt}

% amphibrachic tetrameter
\begin{center}
The code has stopped working with no explanation.\\
It doesn't make sense and you cry in frustration.\\
Just take a deep breath and use these three suggestions,\\
so you will be able to ask the right questions.\\
\end{center}

\section*{Ask yourself:}
\begin{itemize}
\item \textbf{\emph{Why} do you need it to work?}\\
What is the context?
What is the goal?

\item \textbf{How do you \emph{show} that it doesn't work?}\\
What are the steps to reproduce your problem?
How will you know that the problem is resolved?

\item \textbf{What \emph{does} work?}\\
How far are you from a working state?
What similar cases work?
\end{itemize}

Then, \underline{ask for help, without fear!}

\vspace{10pt}

\section*{Frequent Problems?}
Reconsider your work flow.
\begin{itemize}
\item Work on one thing at a time.
\item Always be able to quickly run and verify small, simple cases.
\item Test often.
\item Visualize often.
\item Save working states often.
\item Plan for reproducibility.
\item Make your code visible and useful to others.
\end{itemize}
Version control (e.g. \texttt{git}) is often useful.

\vspace{10pt}
\begin{center}
{\small
By \href{https://www.patricksanan.com}{Patrick Sanan}, 2019.
This document is released under a Creative Commons license and is available at
\href{https://github.com/psanan/it_doesnt_work}{\texttt{github.com/psanan/it\_doesnt\_work}}.
Version 1.0 .\\
\vspace{10pt}
\includegraphics[width=50px]{by-nc-sa.png}
}
\end{center}

\end{document}
